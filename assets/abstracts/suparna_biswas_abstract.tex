% Abstract Example
\subsection*{Estimation of Spectral Risk Measures for Left Truncated and Right Censored Data} % WRITE_HERE
\noindent
\textbf{Speaker:} Dr. Suparna Biswas \\ % WRITE_HERE
\textbf{Affiliation:} WISE Postdoctoral Fellow, Applied Statistics Unit, ISI Bangalore \\ % WRITE_HERE
\textbf{Area of interest:} Statistical Finance \\ 

\noindent\textbf{Abstract.} I will discuss a broader category of risk measures referred to as spectral risk measures, along with their estimation in the context of left truncated and right censored (LTRC) data. LTRC are encountered frequently in insurance loss data due to deductibles and policy limits. Risk estimation is an important task in insurance as it is a necessary step for determining premiums under various policy terms. Spectral risk measures are inherently coherent and have the benefit of connecting the risk measure to the user's risk aversion. We propose a nonparametric estimator of spectral risk measure using the product limit estimator and establish the asymptotic normality for our proposed estimator. We also develop an Edgeworth expansion of our proposed estimator. The bootstrap is employed to approximate the distribution of our proposed estimator and is shown to be second-order accurate. Monte Carlo studies are conducted to compare the proposed spectral risk measure estimator with the existing parametric and nonparametric estimators for left-truncated and right-censored data. Based on our simulation study we estimate the exponential spectral risk measure for two data sets viz; Norwegian fire claims and French marine losses.


