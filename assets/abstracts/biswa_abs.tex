% Abstract Example
\subsection*{Spectral radii for subsets of Hilbert $C^\ast$-modules} % WRITE_HERE
\noindent
\textbf{Speaker:} Biswarup Saha \\ % WRITE_HERE
\textbf{Affiliation:} SRF, Stat-Math Unit, ISI Bangalore\\ % WRITE_HERE
\textbf{Area of interest:} Operator Algebras \\

\noindent\textbf{Abstract.} The notions of joint and outer spectral radii are extended to the setting of Hilbert $C^\ast$-bimodules. A Rota-Strang type characterisation is proved for the joint spectral radius. In this general setting, an approximation result for the joint spectral radius in terms of the outer spectral radius has been established.
This work leads to a new proof of the Wielandt-Friedland's formula for the spectral radius of positive maps. It is observed that algebras generated by tuples of matrices can be determined and their dimensions can be computed by realizing them as linear span of Choi-Kraus coefficients of some easily computable completely positive maps.  This talk is based on a joint work with B V Rajarama Bhat and Prajakta Sahasrabuddhe.

