% Abstract Example
\subsection*{An elementary computational approach to the Jordan-Chevalley decomposition and the Jordan canonical form.} % WRITE_HERE
\noindent
\textbf{Speaker:} Renu Shekhawat \\ % WRITE_HERE
\textbf{Affiliation:} SRF, Stat-Math Unit, ISI Bangalore\\ % WRITE_HERE
\textbf{Area of interest:} Operator Algebras \\

\noindent\textbf{Abstract.} Let $\mathbb K$ be a field, $A$ be a matrix in $M_n(\mathbb K)$, and $\mathbb L$ be the splitting field of the characteristic polynomial of $A$. In this talk, we will prove the existence and uniqueness of the Jordan-Chevalley decomposition of $A$, viewed as a matrix in $M_n(\mathbb L)$, in an elementary computational way. The main strategy is the transformation of $A$ to an appropriate block-diagonal form via similarity transformations obtained from Roth's removal rule. Moreover, if $\mathbb F$ is the fixed field of $\mathrm{Aut}(\mathbb L/ \mathbb K)$ then the potentially-diagonalizable and nilpotent parts of $A$ are both in $M_n(\mathbb F)$; in particular, when $\mathbb K$ is a perfect field, they are both in $M_n(\mathbb K)$. With the block-diagonal form of $A$ in hand, by appropriately using Roth's removal rule in the context of strictly upper-triangular matrices, we will arrive at the Jordan canonical form of $A$.
