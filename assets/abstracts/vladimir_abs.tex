% !TEX root = ../main.tex
% This is a sample file for an abstract submission.
% Author: Indrajit Ghosh
% Created On: Dec 19, 2024
%
% For each abstract submitted by the speakers, create a separate .tex file like this one.
% Then, include these abstract files into the main.tex document.
%
\subsection*{Operator Lipschitz Functions} % CHANGE_HERE
\noindent
\textbf{Speaker:} Professor Vladimir Pellar \\ % CHANGE_HERE
\textbf{Affiliation:} Saint Petersburg State University, Russia \\ % CHANGE_HERE
\textbf{Area of interest:} Mathematical Analysis; and more specifically, Functional Analysis \\

\noindent\textbf{Abstract.} A function \(f\) on the real line is called an operator Lipschitz function if
\[
    \|f(A)-f(B)\|\le \text{const} \, \|A-B\|
\]
for arbitrary self-adjoint operators \(A\) and \(B\).

Not all Lipschitz functions are operator Lipschitz. For example, the function \(|x|\) is not operator Lipschitz.
The class of operator Lipschitz functions plays an important role in perturbation theory.
I am going to give necessary conditions and sufficient conditions for a function to be operator Lipschitz. 
Similarly, one can define operator Lipschitz functions on the circle and on subsets of the complex plane.


