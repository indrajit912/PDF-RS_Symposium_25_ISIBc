% Abstract Example
\subsection*{Expected length of random polynomial lemniscates} % WRITE_HERE
\noindent
\textbf{Speaker:} Dr. Subhajit Ghosh \\ % WRITE_HERE
\textbf{Affiliation:} Postdoctoral Fellow, Stat-Math Unit, ISI Bangalore \\ % WRITE_HERE
\textbf{Area of interest:} Probability and Analysis\\

\noindent\textbf{Abstract.}  A lemniscate of a monic complex polynomial \( Q_n \) is the sublevel set \( \{z \in \mathbb{C} \mid |Q_n(z)| < t\} \) for some \( t > 0 \). Erdős, Herzog, and Piranian initiated the study of lemniscates, focusing on geometric and topological properties such as area, length, and the number of connected components of unit lemniscates (\( t = 1 \)) in their 1958 paper $[1]$. While the maximal length of a lemniscate for a degree-\( n \) polynomial is known to be of the order \( n \), this talk explores the \emph{typical length} of random polynomial lemniscates. Using an integral geometric formula, we show that for polynomials with i.i.d. roots uniformly distributed in \( \mathbb{D} \), the expected length of their lemniscates is bounded above by \( O(\log n) \), significantly less than the optimal order.
\\

\noindent \textbf{Reference.}
\begin{enumerate}[label={[}\arabic*{]}\setlength{\labelsep}{0.5em}, left=0pt, itemsep=0pt]
    \item P.~Erd\H{o}s, F.~Herzog, and G.~Piranian, \textit{Metric properties of polynomials}, J. Analyse Math., 6 (1958), 125--148.
\end{enumerate}
